\documentclass[12pt]{amsart}
\usepackage[margin=1in]{geometry}
%\usepackage{tikz}
%\usetikzlibrary{calc}
\usepackage{amsmath, amsthm, amssymb, graphicx, setspace}
\usepackage{mymacros}
\usepackage{pysyntax}
\usepackage{verbatim}
\usepackage{hyperref}
\usepackage{graphicx}
\usepackage{caption}
\usepackage{subcaption}

\newcommand{\kry}{\mathcal K}
\newcommand{\figref}[1]{\figurename~\ref{#1}}

\title{Hourglass Effect in Granular Materials}
\author{Kevin Joyce}
\date{15 April 2013} 

\begin{document}
\maketitle
\section{Introduction}

The unique physics of granular material flow is distinct from the standard models of 
fluid flow \cite{bell}, and many proposed models are well suited to numerical
simulation. One particularly unique feature that is not captured by standard
fluid models is the so-called ``Hourglass'' effect of granular flow.  That is,
the rate of flow of a granular material through a narrow passage is roughly
constant, independent of the magnitude of the force pushing the material
through. Hence, in the case of an actual hourglass, the amount of sand at the
bottom would be proportional passage of time since the beginning of the flow.

In this paper, we present a simple model that demonstrates this phenomenon in
two dimensions.  We also give a simple explanation for this in terms of
per-particle forces.  We also discuss some specific computational difficulties 
and methods for overcoming them, as well as further applications and improvements.

\section{Methods and Verification} 
This implementation used the \texttt{SciPy} and \texttt{NumPy} \cite{scipy} libraries within the \texttt{Python} programming language.

Our implementation models the granular material as an ensemble of two-dimensional spheres subject to a constant downward gravitational force. The pair-wise interaction between the spheres is governed by a repulsive electrical force (Lennard-Jones) proportional to
\begin{equation}
  \vect f^{e}_{ij} = -\left(\frac{1}{\|\vect r_{ij}\|}\right)^{12}\widehat{\vect r_{ij}},
  \label{ljforce}
\end{equation}
and rotation and non-conservative shearing is approximated by
\begin{equation}
  \vect f^{d}_{ij} = \gamma\, proj_{\vect r_{ij}}(\vect v_{ij}) \, \widehat{\vect r_{ij}}, 
  \label{dforce}
\end{equation}
where $\vect r_{ij}$ is the vector from sphere $i$ to sphere $j$, $\vect
v_{ij}$ is the signed difference of the vector velocities,  $proj_{\vect
r_{ij}}(\vect v_{ij})$ is the component of $\vect v_{ij}$ in the direction of
$\vect r_{ij}$ given by $\widehat{\vect r_{ij}} \cdot \vect v_{ij}$, and hats
indicate normalization. Equation \eqref{dforce} can be thought of as a
first-order approximation the cumulative effect of rotation and
non-conservative shearing modeled by dissipation proportional to velocity.
This is similar to viscous damping.  The parameter $\gamma$ must be selectively
chosen according to the number of particles so that $\vect f^{d}_{ij}$ does not
``over-damp'' and send particles off to infinity.  More accurate models that
derive the rotational and shearing forces are described and implemented in
\cite{cundle} and \cite{bell}.

The walls of the hourglass are modeled by a per-particle Lennard-Jones
electrical force proportional to 
\begin{equation}
  \vect f^w_i = \left(\frac{1}{\|\vect {rw}_{i}\|}\right)^{12}\widehat{ \vect {rw}_i }
\label{wforce}
\end{equation}
where $\vect{rw}_{i}$ is the distance from sphere $i$ to the wall. This
distance is calculated in the standard way by minimizing the point-wise
distance for each point on the wall.  In the case of straight line $ax + by +
c=0$, this is given by 
$$
  \|\vect{rw}_i \| = \frac{|ax + by +c|}{\sqrt{a^2 + b^2}}.
$$
and $\widehat{\vect{rw}_i}$ points perpendicular to the slope of the line.
That is $a\vect i + b\vect j$ normalized.  Each sphere-sphere and line-sphere
force is only calculated when their respective distances are with in Lennard
Jones radius of $2^{1/6}$.  Note that under this constraint, the geometry of
the system dictates that the average interaction is much less than a full pair
by pair comparison.  This reduces the pair-wise force calculations from
$O\left(N^2\right)$ to $O(N)$ ($N$ spheres).  A similar argument reduces 
the sphere-wall force calculation.  This can be implemented via a sparse
matrix or list of neighbors data type.

\begin{figure}[h]
\includegraphics[width=.6\textwidth]{particle_flux.pdf}
\caption{ This plot gives the flux of particles through the horizontal slice between $y=8$ and $y=10$. } 
\label{flux}
\end{figure}

Animations of the simulation provide verification of the model and a rubric for
choosing $\gamma$.  We also estimate the flow out of the hour glass calculating
the vertical flux of particles just below the opening.  This is shown in
\figref{flux} and provides evidence of the constant flow phenomenon being
correctly modelled by our simulation.

\section{Results and Analysis }
\begin{figure}[h]
\begin{subfigure}{.4\textwidth}
\includegraphics[width=\textwidth,height=.14\textheight]{hourglass_200.pdf}
\includegraphics[width=\textwidth,height=.14\textheight]{hourglass_400.pdf}
\includegraphics[width=\textwidth,height=.14\textheight]{hourglass_600.pdf}
\includegraphics[width=\textwidth,height=.14\textheight]{hourglass_800.pdf}
\includegraphics[width=\textwidth,height=.14\textheight]{hourglass_1000.pdf}
\includegraphics[width=\textwidth,height=.14\textheight]{hourglass_1200.pdf}
\includegraphics[width=\textwidth,height=.14\textheight]{hourglass_1400.pdf}
\end{subfigure}
\begin{subfigure}{.55\textwidth}
\includegraphics[width=\textwidth,height=.14\textheight]{component_slice_forces_200.pdf}
\includegraphics[width=\textwidth,height=.14\textheight]{component_slice_forces_400.pdf}
\includegraphics[width=\textwidth,height=.14\textheight]{component_slice_forces_600.pdf}
\includegraphics[width=\textwidth,height=.14\textheight]{component_slice_forces_800.pdf}
\includegraphics[width=\textwidth,height=.14\textheight]{component_slice_forces_1000.pdf}
\includegraphics[width=\textwidth,height=.14\textheight]{component_slice_forces_1200.pdf}
\includegraphics[width=\textwidth,height=.14\textheight]{component_slice_forces_1400.pdf}
\end{subfigure}
\caption{ This matrix of plots gives a screen shot of the animation at various
time points to the left of the average (at a given height) accelerations of the
particles component-wise. }
\label{animation}
\end{figure}

Snapshots of the animation along side the components of force averaged
per-sphere horizontally at various heights of the hourglass are given in
\figref{animation}.  Note the spikes of horizontal-component force (column 2)
occur at nearly the same height with large net positive spikes of
vertical-component force (column 3).  This suggests that the force due to
gravity of the granular material above heights where the spikes occur is
countered by a network of forces from sphere to sphere just above the opening
that results in a cumulative upward force.  This upward force  accounts for
``jamming'' of the particles just above those nearest to the opening.
Perturbations break the jamming but another jam is quickly reformed.  This is
indicated in \figref{jam}.  This figure indicates the magnitude of the average
vertical component at various heights at each time step.  Note the spikes
indicating jams followed soon after by negative vertical force indicating a jam break.
Note also that the largest magnitude upward force is generally at $y=10$ and $y=11$,
indicating that particles just at the opening are free to fall.  That is, all 
of the force due to gravity above the lightly colored regions is \emph{not} transferred
to just at the opening below the lightly colored regions. 
\begin{figure}[h]
\includegraphics[width=.7\textwidth]{jam_show.pdf}
\caption{ Average vertical component of force over horizontal cross-sections of the hourglass at each integration step. }
\label{jam}
\end{figure}
\section{Conclusion}

%Our simulation agrees with the results in \cite{ringlein}, in that we have observed a positive relationship between molecular surface area and the force of static friction (linear increase of intercepts in \figref{forceload}). This relationship appears to be linear based on the three linear increase of surface area in the three simulations.  This confirms the implied intuitin in the introduction that implies a model of static friction contrary to Amonton's Law (2).  We also observed a linear relationship between an applied normal force and the force of static friction (fit of points to line in \figref{forceload}). These results together suggest that a more appropriate model for the force of molecular static friction is
%$$
%  F_s = \mu_s W + c A,
%$$ 
%where $W$ is the applied normal force,  $A$ is the surface area of contact, and $\mu_s$ and $c$ are constants of proportionality.  In our simulation we predicted these vales to be $\hat \mu_s = 0.3767$ and $\hat c = 1.0$ by averaging predicted slopes and fitting a line through the intercepts of lines in \figref{forceload}.
%
%The next question is why $c = 0$ on a macroscopic scale, reconciling with
%Amontons' original model of friction.  This question is addressed in
%\cite{ringlein}, and they attribute the lack of area dependence to the fact
%that macroscopic surfaces are quite ``rough'' and there is relatively sparse
%area of actual contact between atoms.  This reconciles with intuition regarding
%adhesion, in that adhesion becomes important when materials easily deform as in tape
%or rubber \cite{ringlein}.  
%
%A future model for this situation might
%be to model the contact surfaces in a probabilistic way so that local maxima
%are randomly distributed on each contact surface.  Although our model codes
%allows for three dimensional particle simulations, it does not scale well
%beyond about 256 particles due to the distance calculation.  Future
%optimization may make this simulation possible.
%
%

\begin{thebibliography}{[1]}
  \bibitem{cundle} Cundall, P. A. and Strack D. L. \emph{A discrete numerical model of granular assemblies}. G\'eotechnique 1979, No. 1, 47-65
  \bibitem{bell} Bell, N., Yu Y. and Mucha, P. J. \emph{Particle-Based Simulation of Granular Materials}.  Eurographics/ACM SIGGRAPH Symposium on Computer Animation 2005.
  \bibitem{scipy} Eric Jones and Travis Oliphant and Pearu Peterson and others, \emph{{SciPy}: Open source scientific tools for {Python}}, 2001--, \url{http://www.scipy.org/}.
  \bibitem{matplotlib} Hunter, J.D. \emph{Matplotlib: A 2D Graphics Environment}, Computing in Science \& Engineering, Vol. 9, No. 3. (2007), pp. 90-95.
\end{thebibliography}
\end{document}
